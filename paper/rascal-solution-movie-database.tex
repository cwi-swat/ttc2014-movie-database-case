\documentclass[submission,copyright,creativecommons]{eptcs}
\providecommand{\event}{TTC 2014} % Name of the event you are submitting to
\usepackage{breakurl}             % Not needed if you use pdflatex only.
\usepackage{rascal}
\usepackage[T1]{fontenc}
\usepackage[scaled=0.8]{beramono}
\title{The TTC 2014 Movie Database Case: Rascal Solution\thanks{This
    research was supported by the Netherlands Organisation for
    Scientific Research (NWO) Jacquard Grant ``Next Generation
    Auditing: Data-Assurance as a service'' (638.001.214).}}
\author{Pablo Inostroza \and Tijs van der Storm}
\def\authorrunning{Inostroza and Van der Storm}
\def\titlerunning{TTC'14: Rascal}

\begin{document}
\maketitle

\begin{abstract}
Rascal is meta programming language for processing source code in the broad sense (models, documents, formats, languages, etc.). In this short note we discuss the implementation of the ``TTC'14 Movie Database Case'' in Rascal. In particular we will highlight the challenges and benefits of using a functional programming language for transformation what is, in essence, a large graph. 
\end{abstract}

\section{Introduction}

Rascal is a meta programming language for source code analysis and transformation~\cite{Rascal,RascalGTTSE}. 
Concretely, it is targeted at analyzing and processing any kind of ``source code in the broad sense''; this includes importing, analyzing, transforming, visualizing and generating, models, data files, program code, documentation etc.
 
Although Rascal features a Java-like syntax, it is a functional programming in that all data is immutable (implemented using persistent data structures), and function programming concepts are used throughout: algebraic data types, pattern matching, higher-order functions, comprehensions etc. 

Specifically for the domain of source code manipulation, however, Rascal features powerful primitives for parsing (context-free grammars), traversal (visit statement), relational analysis (transitive closure, image etc.), and code generation (string templates). 
The standard library includes programming language grammars (e.g., Java), IDE integration
with Eclipse, numerous importers (e.g. XML, CSV, YAML, JSON etc.) and a rich visualization framework. 

\subsection{Representing IMDB in Rascal}

Since Rascal is a functional programming language, where all data is immutable, the IMDB models have to be represented as trees instead of graphs. If there are cross references in the model, these have to be represented using (symbolic or opaque) identifiers which can be used to look up referenced elements. 

The following data type captures the structure of the IMDB models:
%\CAT{Keyword}{alias} Id = \CAT{Keyword}{int};
\begin{rascal}
\CAT{Keyword}{data} IMDB   = imdb(\CAT{Keyword}{map}{}[Id, Movie] movies, \CAT{Keyword}{map}{}[Id, Person] persons, 
                   \CAT{Keyword}{set}{}[Group] groups, \CAT{Keyword}{rel}{}[Id movie, Id person] stars);
\CAT{Keyword}{data} Movie  = movie(\CAT{Keyword}{str} title, \CAT{Keyword}{real} rating, \CAT{Keyword}{int} year);
\CAT{Keyword}{data} Person = actor(\CAT{Keyword}{str} name) | actress(\CAT{Keyword}{str} name);
\CAT{Keyword}{data} Group  = couple(\CAT{Keyword}{real} avgRating, Id p1, Id p2, \CAT{Keyword}{set}{}[Id] movies);
\end{rascal}

An IMDB model is constructed using the \texttt{imdb} constructor. It contains the set of movies, persons, groups and a relation \texttt{stars} encoding which movie stars which persons. 
Both movies and persons are identified using the opaque \texttt{Id} type. 
To model this identification, the \texttt{movies} and \texttt{persons} field of an IMDB model are maps from such identifiers to the actual movie resp. person. 

Movies and persons are simple values containing the various fields that pertain to them. 
The \texttt{Group} type captures couples as required in Task 2. A \texttt{couple} references two persons and a set of movies using the opaque identifiers \texttt{Id}. 



\section{Task 1: Synthesizing Test Data}

Synthesizing test data creates values of the type \texttt{IMDB} as declared in the previous section. The process starts with an empty model (\texttt{imdb((), (), \{\}, \{\})}\footnote{The syntax \texttt{()} indicates an empty map.}), and then consecutively merges it with test models for a value in the range $1,...,n$.
Each test model in turn consists of merging the negative and positive test model as discussed in the assignment. 
As an example, below we list the creation of the positive test model:
\begin{rascal}
IMDB createPositive(\CAT{Keyword}{int} i) = imdb(movies, people, {}, stars)
    \CAT{Keyword}{when} movies := ( j: movie(\CAT{Constant}{"m\textless{}}j\CAT{Constant}{\textgreater{}"}, toReal(j), 2013) | j \textless{}- {}[10*i..10*i+5] ),
         people := ( 10*i: actor(\CAT{Constant}{"a\textless{}}10*i\CAT{Constant}{\textgreater{}"}), 10*i+1: actor(\CAT{Constant}{"a\textless{}}10*i+1\CAT{Constant}{\textgreater{}"}), 
                     10*i+2: actor(\CAT{Constant}{"a\textless{}}10*i+2\CAT{Constant}{\textgreater{}"}), 10*i+3: actress(\CAT{Constant}{"a\textless{}}10*i+3\CAT{Constant}{\textgreater{}"}), 
                     10*i+4: actress(\CAT{Constant}{"a\textless{}}10*i+4\CAT{Constant}{\textgreater{}"}) ),
         stars := \{{}\textless{}10*i, 10*i\textgreater{}, \textless{}10*i, 10*i+1\textgreater{}, \textless{}10*i, 10*i+2\textgreater{}, \textless{}10*i, 10*i+3\textgreater{},
                 \textless{}10*i+1, 10*i\textgreater{}, \textless{}10*i+1, 10*i+1\textgreater{}, \textless{}10*i+1, 10*i+2\textgreater{}, \textless{}10*i+1, 10*i+3\textgreater{},
                 \textless{}10*i+2, 10*i+1\textgreater{}, \textless{}10*i+2, 10*i+2\textgreater{}, \textless{}10*i+2, 10*i+3\textgreater{},
                 \textless{}10*i+3, 10*i+1\textgreater{}, \textless{}10*i+3, 10*i+2\textgreater{}, \textless{}10*i+3, 10*i+3\textgreater{}, \textless{}10*i+3, 10*i+4\textgreater{},
                 \textless{}10*i+4, 10*i+1\textgreater{}, \textless{}10*i+4, 10*i+2\textgreater{}, \textless{}10*i+4, 10*i+3\textgreater{}, \textless{}10*i+4, 10*i+4\textgreater{}\}{};          
\end{rascal}

The function uses map comprehensions to create the \texttt{movies} and \texttt{people} fields, and a binary relation literal to create the \texttt{stars} relation. It then simply returns a values containing all those fields.



\section{Task 2: Adding Couples}

Task 2 consists of enriching IMDB models with ``couples'': pairs of persons that performed in the same movie, once or more often. This transformation is expressed by updating the \texttt{couples} field with the result of the following function:

\begin{rascal}
\CAT{Keyword}{set}{}[Group] makeCouples(IMDB m) \{{}
    costars = toMap(m.stars); couples = ();
    \CAT{Keyword}{for} (mov \textless{}- costars, s1 \textless{}- costars{}[mov], s2 \textless{}- costars{}[mov], s1 \textless{} s2) 
       couples{}[\textless{}s1, s2\textgreater{}]?\{\} += \{{}mov\}{};
    \CAT{Keyword}{return} \{{} couple(0.0, x, y, ms) 
                     | \textless{}x, y\textgreater{} \textless{}- couples, ms := couples{}[\textless{}x, y\textgreater{}], size(ms) \textgreater{}= 3 \}{};
\}{}
\end{rascal}

This function first convert the binary relation \texttt{stars} to a map from movie Id to set of person Ids. The central for loop iterates over all movies and all combinations of two actors and adds the movie to a table maintaining the set of movies for all couples (a map taking tuples of person Ids to sets of movie Ids). The side condition \texttt{s1 \textless s2} ensure we don't visit duplicate or self combinations.
The question mark notation initializes a map entry with a default value, if the entry did not yet exist.
In the final statement, a set of Groups is returned containing all couples which performed in 3 or more movies.




\section{Task 3: Computing Average Ratings for Couples}

As can be seen in the previous section, the average rating field of couples is initialized to 0.0. In this task we again transform an IMDB model, this time enriching each couple with its average rating of the movies the couple co-starred in. 

The following function performs this transformation:
\begin{rascal}
IMDB addGroupRatings(IMDB m) = m{}[groups=gs]
  \CAT{Keyword}{when} gs :=
    \{{} g{}[avgRating = toReal(mean(ratings))] 
       | g \textless{}- m.groups, ratings := {}[ m.movies{}[x].rating | x \textless{}- g.movies ]\}{};
\end{rascal}

The groups field of the model \texttt{m} is updated with a new set of groups, as created in the when-clause of the function. 
The new set of groups is created using a comprehension, updating the \texttt{avgRating} field of each group. 
The average is computed based on the list of ratings obtained from the movies contained in \texttt{m} that are referenced in the group \texttt{g}. 

\section{Extension Task 1: Top 15 Rankings}

\begin{rascal}
\CAT{Keyword}{alias} Ranking = \CAT{Keyword}{lrel}{}[\CAT{Keyword}{set}{}[Person], \CAT{Keyword}{real}, \CAT{Keyword}{int}]; 

Ranking rank(\CAT{Keyword}{int} n, IMDB m, \CAT{Keyword}{bool}(Group, Group) gt) =
    take(n, 
       {}[\textless{}\{{}m.persons{}[x] | x \textless{}- getPersons(g)\}{}, g.avgRating, size(g.movies)\textgreater{} 
            | Group g \textless{}- sort(m.groups, gt)]);
\end{rascal}    


\begin{rascal}
Ranking top15avgRating(IMDB m)    = rank(15, m, greaterThan(getRating));    
Ranking top15commonMovies(IMDB m) = rank(15, m, greaterThan(getNumOfMovies));
\end{rascal}

greaterThan, getRating, getNumOfMovies.

\section{Extension Task 2: Generalizing groups to cliques}

\begin{rascal}
\CAT{Keyword}{data} Group = clique(\CAT{Keyword}{real} avgRating, \CAT{Keyword}{set}{}[Id] persons, \CAT{Keyword}{set}{}[Id] movies);
\end{rascal}


\begin{rascal}
\CAT{Keyword}{set}{}[Group] makeCliques(IMDB m, \CAT{Keyword}{int} n) \{{}
    costars = toMap(m.stars); cliques = ();
    \CAT{Keyword}{for} (Id movie \textless{}- costars, \CAT{Keyword}{set}{}[Id] s \textless{}- combinations(costars{}[movie], n))
      cliques{}[s]?\{\} += \{{}movie\}{};
    \CAT{Keyword}{return} \{{}clique(0.0, s, ms) | s \textless{}- cliques, ms := cliques{}[s], size(ms) \textgreater{}= 3 \}{};
\}{}
\end{rascal}

\paragraph{Extension Task 3 \& 4}

These are the same as Task 3 and Extension of Task 1 respectively,
because the code is polymorph over groups.

\section{Concluding Remarks}

The tasks were easy to implement, with little code. The size of the
implementation is around 130 SLOC, including some helper functions,
but excluding loading the model from XML which is another 38 SLOC.

Rascal's module system showed its benefits; some tasks could be
implemented as modular extensions of earlier tasks, combining
extension of data types (Extension Task 2) and extension of functions
(Extension Tasks 3 and 4).

Performance: extracting cliques $> 2$ takes longer than 5 minutes on a
3.3Mb IMDB file. The most probable reason is that
\texttt{combinations} is slow, even though we use a dynamic
programming algorithm; it is possible that immutability works against
us here.

\section{Bibliography}

\nocite{*}
\bibliographystyle{eptcs}
\bibliography{generic}
\end{document}
